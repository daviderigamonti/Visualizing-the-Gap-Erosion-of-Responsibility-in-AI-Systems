\section{Conclusions}\label{sec:conclusions}

% Recap
In this paper we discussed about the various intricacies of the term responsibility and its alternative formulations, while mentioning some of the traditional ethical frameworks to manage responsibility and their philosophical grounds.
We introduced the notion of \textit{responsibility gap} and we observed the fact that classical engineering frameworks inevitably fall short when dealing with autonomous systems, which can be considered the core point of this paper.
Once we recognized the existence of the responsibility gap, we turned our attention to potential solutions, analyzing and comparing various ideas present in the literature.
Lastly, we took into account some counterarguments that challenged the feasability of sociotechnical solutions and questioned the actual existence of a responsability gap.

% Conclusions
In conclusion, addressing the responsibility gap is not an easy task, and acknowledging its presence in modern society is the first step towards the creation of a new responsibility ecosystem for autonomous system.
There needs to be a collective effort to face the problem in the correct way while it is still possible to make changes to regulations and standards without destabilizing society.
Autonomous systems are becoming more widespread and pervasive with each day that passes, and soon enough there might come a time where we will need to come to terms with the choices made in the past, at that point the presence of a solid foundation for responsibility management in learning systems will be of utmost importance.