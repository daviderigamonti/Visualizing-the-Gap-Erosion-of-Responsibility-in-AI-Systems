\section{Contrasting Viewpoints}\label{sec:viewpoints}

% Introduction
Through a comprehensive analysis of contrasting viewpoints we can shed light on the issues of our thesis and hopefully reach solid conclusions.

% Deflationism evolution into recognizing the gap
One of the earlier formulations of \textit{deflationism} \parencite{HAYTMP} that were analyzed in the previous section clearly goes against our main thesis as it refuses to acknowledge the responsibility gap as meaningful.
However, the deflationist ideology evolved over time to accept the notion of responsibility gap and focus on other aspects of its original theory \parencite{SIJWA}; this result grants some empirical credibility to the foundation of our thesis.

% Legal counterargument
A possible counterargument to the main thesis is that trying to construct a new framework with new rules is superfluous as the classical one can be adapted to accomodate autonomous systems with the help of the law's inherent flexibility when interpreted by a court.
The previous claim makes some strong assumptions on the quality, effectiveness and homogeneity of law systems around the world and across different cultures, however our main thesis still holds since we would like to approach the issue on an ethical level of thought, where legal liability is only a part of the bigger picture.

.\todo{write}