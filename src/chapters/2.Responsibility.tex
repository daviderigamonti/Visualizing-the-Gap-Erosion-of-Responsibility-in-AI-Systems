\section{Analyzing Responsibility}\label{sec:responsibility}

The term \textit{responsibility} is often used as a broad expression to convey the notion of ``someone'' (which can be a person, an institution, a corporation or more generally, an \textit{agent}) having the duty of upholding certain expectations defined by another (or the very same) agent towards a given goal; however, it's important to acknowledge the presence of derived terms that represent different ``flavors'' of \textit{responsibility}, in this section we focus on the specific expressions that are most relevant to this paper, nevertheless, it's important to note that numerous taxonomies can be formulated.

In the Assessment List for Trustworthy Artificial Intelligence (ALTAI) for self-assessment, \textit{accountability} is defined as a term which refers to the idea that one is responsible for their actions and consequences, therefore they must be able to explain their aims, motivations, and reasons (\cite{ALTAI} as cited in~\cite{NOVAIA}); \todo{too many `of's} of particular importance is the presence of an autorithy in charge of supervising the conduct of the agent held accountable and thus, sitting closely to the definition of \textit{answerability} \parencite{NISAIA} in contrast with the notion of \textit{moral responsibility}, which assumes an internal analysis against the very own moral values of the agent of interest.
An important distinction that needs to be made is the difference between \textit{active responsibility} and \textit{passive responsibility} \parencite{ETE}: while the first term is appropriate for addressing the continuous and preemptive effort that one must make to care about a certain goal while it is being attained, the second is applicable in the event that something undesirable has already happened instead and can be subdivided in \textit{accountability}, \textit{blameworthiness} (indirect responsibility) and \textit{liability} (economical/legal responsibility) according to \parencite{ETE}.

.\todo{Add Feinberg's responsibility model and why it doesn't apply}
.\todo{Consider moving Feinberg's model applicabilityto section 3}
