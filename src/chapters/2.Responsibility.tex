\section{Analyzing Responsibility}\label{sec:responsibility}

% Acknowledging responsibility variations
The term \textit{responsibility} is often used as a broad expression to convey the notion of ``someone'' (which can be a person, an institution, a corporation or more generally, an \textit{agent}) having the duty of upholding certain expectations defined by another (or the very same) agent towards a given goal.
It is crucial to acknowledge the presence of derived terms that represent different ``flavors'' of \textit{responsibility}; in this section, our focus will be on the specific expressions that are most relevant to this paper.
Nevertheless, it is also important to note that numerous taxonomies can be formulated.

% Main responsibility variations
In the Assessment List for Trustworthy Artificial Intelligence (ALTAI) for self-assessment, \textit{accountability} is defined as a term which refers to the idea that one is responsible for their actions and consequences, therefore they must be able to explain their aims, motivations, and reasons (\cite{ALTAI} as cited in~\cite{NOVAIA}); it is essential to note the presence of an authority in charge of supervising the conduct of the agent held accountable and thus, closely aligning with the definition of \textit{answerability} \parencite{NISAIA} in contrast with the notion of \textit{moral responsibility}, which assumes an internal analysis against the very own moral values of the agent of interest.
In addition, it is also possible to identify the \textit{patient of responsibility} figure as someone that is affected by the actions of the agent and is entitled to demand accountability for those actions \parencite{COEAIR}.

% Active / passive responsibility + role responsibility
An important distinction that needs to be made is the difference between \textit{active responsibility} and \textit{passive responsibility} \parencite{ETE}: while the first term is appropriate for addressing the continuous and preemptive effort that one must make to care about a certain goal while it is being achieved, the second is applicable in the event that something undesirable has already happened, in case of which it can be subdivided in \textit{accountability}, \textit{blameworthiness} (indirect responsibility) and \textit{liability} (economical/legal responsibility) according to \parencite{ETE}.
Another important aspect of responsibility is its context, as the notion of \textit{role responsibility} suggests.
In our daily life we partake in social roles which come with their respective responsibilities that may be conflicting; specifically, \textit{professional responsibility}, is a particular type of role responsibility which concerns the professional life of an agent \parencite{ETE}.

% Feinberg's / Aristotle responsibility interpretation
Feinberg identifies two main causes for an agent to be held morally blameworthy for any given harm: \textit{causality} (the agent's actions contributed in causing the harm) and \textit{faultiness} (the actions were intentionally harmful or a result of negligent behavior) \parencite{FEISC}.
Another viable interpretation is the aristotelian one, which specifies the \textit{control condition} (intent and freedom of action) and the \textit{epistemic condition} (awareness or non-ignorance) \parencite{FISRAC}.

% Example
It is easy to see how these frameworks fit quite well with most engineering tasks due to the fundamental instrumentality of the produced artifacts.
For example, in the field of weapon systems, we can see that if the computerized aiming aid of a mortar happened to point the armament towards the wrong direction, and this happened due to an error in the code dedicated to the calculation of coordinates, once the root cause is assessed, it wouldn't be unthinkable to ascribe the responsibility to the programmer that wrote the code (provided that there were no other failures within the system, that the error was imperceptible to a human supervisor and that the calculations performed by the system don't involve machine learning or similar techniques).
This happens because the code that runs on the system is considered an artifact with a precise function to carry out; the programmer knows this function and writes code that specifically performs the given task.
If the artifact presents incorrect behavior it is reasonable to assume that its designer should be held accountable for the possible damages, thus resulting in the fulfillment of the \textit{faultiness} considering their expertise and potential to anticipate such misbehavior.

